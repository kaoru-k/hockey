\documentclass{jarticle}
\usepackage[dvipdfmx]{graphicx}
\usepackage{here}
\usepackage{url}
\usepackage{ascmac}
\usepackage[top=30truemm,bottom=30truemm,left=25truemm,right=25truemm]{geometry}
\title{GitHub入門①}
\author{木村 薫}
\begin{document}
\maketitle
\section{はじめに}
GitHub入門①では,既存のリポジトリをインポートし,変更を反映させることが目標である.リポジトリ作成などは含まれていないのであしからず.
%
\section{用語解説}
\begin{description}
\item[リポジトリ]\mbox{}\\
  プログラムやソースコード,リソース(画像・音声ファイル)などを保管している場所.
\item[GitHub]\mbox{}\\
  リポジトリをホスティング,つまりサーバーを提供しているサイト.無料で使えるが,機能は限られる.
\item[Git]\mbox{}\\
  バージョン管理システム.最近はsubversionよりも使われることが多いらしい.
\end{description}
%
\section{準備}
ユーザ名とメールアドレスを設定しよう\\
\begin{itembox}[l]{ユーザ名とメールアドレスを設定}
\begin{verbatim}
git config --grobal user.name "[名前]"
git config --grobal user.email "[メールアドレス]"
\end{verbatim}
\end{itembox}
\section{ローカルリポジトリを取得する}
まず,ディレクトリを作って,そこに移動しよう\\
\begin{itembox}[l]{ディレクトリをつくる}
\begin{verbatim}
mkdir [directory]
cd [directory]
\end{verbatim}
\end{itembox}
\\
リモートリポジトリを取得する.\\
\begin{itembox}[l]{リポジトリを取得する}
{\tt git clone [url]}
\end{itembox}
今回の場合,[url]は\url{https://github.com/kaoru-k/hockey.git}.\\
\\
これで,初期設定は終わり.
%
\section{ローカルの変更をリモートリポジトリに反映させる}
ディレクトリ内で,ファイルを作成したり,ソースコードを編集した場合には,それをリモートリポジトリ(GitHub上のリポジトリ)に反映しなければならない.その手順についての説明.
\subsection{ファイルやディレクトリをインデックスに追加}
ファイルやフォルダを作成したらインデックスに追加しよう.追加しないと,編集が反映されない.
\begin{itembox}[l]{ファイルやディレクトリをインデックスに追加}
{\tt git add [ファイル名]}\\
もしくは\\
{\tt git add .}
\end{itembox}
\\
追加ができたらcommitできるようになる.
\\
\subsection{変更をリポジトリに書き込む}
編集がおわったらリポジトリに変更を書きこもう.\\
\begin{itembox}[l]{変更をリポジトリに書き込む}
  {\tt git commit -am "[comment]"}
\end{itembox}
[comment]には,変更内容についてコメントを入れること.例えば「〜.cの関数○○○を編集した」など.
\\
%
\subsection{リモートリポジトリに反映させる}
commitした内容をリモートリポジトリに反映する.\\
\begin{itembox}[l]{ローカルのリポジトリの内容をリモートに反映させる}
  {\tt git push origin master}
\end{itembox}
この時,GitHubのユーザ名とパスワードを尋ねられるので答える(入力を省略する方法は後述).\\
これで,自分が編集した内容がリモートに反映された.commitとpushは頻繁にしよう.
%
\subsection{リモートの変更をローカルリポジトリに反映する}
他人の編集を取り込むにはこれを実行する\\
\begin{itembox}[l]{リモートの変更を取り込む}
{\tt git pull}
\end{itembox}
%
\subsection{ファイル・ディレクトリの削除・名前変更}
Gitは賢いので,名前を変えても気づいてくれるようだが,一応正しい方法を説明しておく.\\
\begin{itembox}[l]{ファイル・ディレクトリの削除}
ファイルの場合は\\
{\tt git rm [filename]}\\
\\
ディレクトリの場合は\\
{\tt git rm -r [directory]}\\
\end{itembox}
\\
\begin{itembox}[l]{ファイルのリネーム}
{\tt git mv [変更前] [変更後]}
\end{itembox}
\end{document}
  
